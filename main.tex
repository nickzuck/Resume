%----------------------------------------------------------------------------------------
%	PACKAGES AND OTHER DOCUMENT CONFIGURATIONS
%----------------------------------------------------------------------------------------
\documentclass[margin, centered]{res}
\topmargin=-0.5in
\oddsidemargin -.5in
\evensidemargin -.5in
\textwidth=6.5in
\itemsep=0in
\parsep=0in
\newsectionwidth{1in}
\usepackage[pdftex]{graphicx}
\usepackage{enumitem}
\usepackage{wrapfig}
\usepackage{helvet}
\usepackage[colorlinks = true,
            linkcolor = blue,
            urlcolor  = blue,
            citecolor = blue,
            anchorcolor = blue]{hyperref}
\setlength{\textwidth}{6.5in} % Text width of the document
\setlength{\textheight}{720pt}

\begin{document}

%----------------------------------------------------------------------------------------
%	NAME AND ADDRESS SECTION
%----------------------------------------------------------------------------------------\
\begin{center}
    \hspace{-\hoffset}
    \huge\bf{Nikhil Kumar Singh}
\end{center}
\vspace{-4mm}
\moveleft\hoffset\vbox{\hrule width 19cm height 0.5pt}
\vspace{-9mm}
\begin{center}
    \hspace{-\hoffset}
    \href{mailto:nikhil1945.singh@gmail.com}{nikhil1945.singh@gmail.com} ~\textbullet~ \(+91\) 9818289841 ~\textbullet~ \#I-1113, Harinagar Extn., Badarpur, New Delhi, India
\end{center}
\vspace{-7mm}
\begin{resume}

%----------------------------------------------------------------------------------------
%	EDUCATION SECTION
%----------------------------------------------------------------------------------------
\section{Education}
\textbf{B.Tech in Computer Science and Engineering} \hfill 2013 - 2017 \\
\href{http://ipu.ac.in/}{Guru Gobind Singh Indraprastha University, Delhi, India}
\begin{itemize}
 \item Percentage -  \textbf{80.8} (June 2016)
\end{itemize}
\textbf{Secondary School} - Kalka Public School, Delhi (CBSE) - 78.8\% \hfill 2012-2013\\
\textbf{High School} - St. Johns School, Faridabad (CBSE) - 8.6 CGPA \hfill 2010-2011\\
 
%---------------------------------------------------------------------------------------
%	EXPERIENCE SECTION
%----------------------------------------------------------------------------------------
\section{Experience}
\textbf{Software Development Intern, \href{http://courses.geeksforgeeks.org}{GeeksforGeeks, Noida}} \hfill June, 2016 - Aug, 2016\\
Created a platform to host courses for GeeksforGeeks. The django application allows the moderators and contributors of the coures to easily add the content from the existing sources as well as to create a new content. The application allows the content curators to arrange the course material in an organised way. The users can track their progress of particular topic on GeeksforGeeks easily with the help of Courses.  \\
\\
\textbf{Python Content Curator, \href{http://geeksforgeeks.org}{GeeksforGeeks, Noida}} \hfill Jan, 2016\\
\begin{itemize}
  \item Converted the existing C++ codes of Tree and Linked List Data Structures to Python
  \item Articles on Basics of  \href {http://www.geeksforgeeks.org/python/}{Python}
  \item Codes in Python for function problems on \href {http://practice.geeksforgeeks.org}{Practice | GeeksforGeeks}
\end{itemize}
\\
\textbf{Backend Developer, \href {http://shoesonloose.com/}{ShoesOnLoose, Delhi}} \hfill Oct, 2015 - Dec, 2015 \\
%\emph{Mentored by \href{http://en.wikipedia.org/wiki/Deepak_B._Phatak}{Prof. Deepak B Phatak, Dept. of CSE, IIT Bombay}} \hfill May, 2014 - July, 2014 \\
Developed the Backend for customized travel itinerary of a Trip. The user can use this application to customize their travel without contacting the salesperson, and the final cost will be automatically adjusted. With the help of this application the user can add location, activities and extra days to their existing travel plan as well as change the hotels and travel details as per their needs. \\
\\
\textbf{Summer Intern, \href{http://www.ilabs.co}{iLabs.co, Delhi}} \hfill June, 2015 - July, 2015 \\
Training in Web Development with Ruby on Rails as well as Backend Development of the website for the startup named \href {http://competifyin.herokuapp.com} {Competify} based on Ruby on Rails. \\
%----------------------------------------------------------------------------------------
%	TECHNICAL SKILLS SECTION
%----------------------------------------------------------------------------------------
\section{Technical \hspace{2mm} Skills}
\textbf{Strongest Areas} - Data Structures, Algorithms, Software Engineering \\
\textbf{Languages} -
\begin{itemize}
	\item \textbf{Proficient} - Python, C++
	\item \textbf{Intermediate} - Ruby, HTML, CSS, MySQL
	\item \textbf{Novice} - Java, Javascript
\end{itemize} 
\textbf{Tools/Frameworks} -
\begin{itemize}
	\item \textbf{Proficient} - Git, Django, web2py
	\item \textbf{Intermediate} - Ruby on Rails, Beautiful Soup
	\item \textbf{Novice} - scikit-learn
\end{itemize} 
%---------------------------------------------------------------------------------------
%	PUBLICATION SECTION
%---------------------------------------------------------------------------------------

%\section{Publications}
%\begin{itemize}[leftmargin=*]
%\item Devanshu J, \textbf{Ashish K}, Rakshit S, Sameer S, ``Recommendation Techniques for Adaptive E-learning'', Advances in Computer Science and Information Technology, vol. 2, No. 1, 2015. \href{https://drive.google.com/file/d/0B6A-3_6rwie9bS1OaFdzbW9BZXM/view?usp=sharing}{view here}
%\item \textbf{Ashish Kedia} and Anusha Prakash, "Data Synchronization on Android Clients", International Conference on Communication Software and Networks, June 6-7$^{th}$, 2015, Chengdu, China. \href{http://ieeexplore.ieee.org/xpl/articleDetails.jsp?reload=true&arnumber=7296156}{view here}
%\end{itemize}

%----------------------------------------------------------------------------------------
%	Selected Projects Section
%----------------------------------------------------------------------------------------
\section{Projects}
% All projects available on git : \url{https://www.github.com/ashish1294}
%\setlist[itemize]{
\begin{itemize}[leftmargin=*]
 \item \textbf{Visual Category Recognition using SVM-KNN} : A computer vision based project in python which predicts the category of the given object on \href {https://www.vision.caltech.edu/Image_Datasets/Caltech101/}{Caltech 101 Dataset}. This project uses the combination of both the techniques (SVM and KNN) to predict the actual label of the image.
 \item \textbf{\href{https://github.com/nickzuck}{Codechef and SPOJ Rank Fetcher}} - Developed a script in Python using web scraping to find the all the ranks of all the users whom you care about. You are only required to enter the valid codechef or spoj usernames in a text file and then run the script from the terminal.
 \item \textbf{Facebook Photo Fetcher} : Wrote a python script to fetch all the photos of particular page from a given data. It uses Facebook Graph API to fetch the photos.
 \
\end{itemize}

%----------------------------------------------------------------------------------------
%	ACHIEVEMENT SECTION
%----------------------------------------------------------------------------------------

\section{Achievements and Awards}
\begin{itemize}[leftmargin=*]
 \item Selected in ACM-ICPC regionals at IIT Kharagpur and Amritapuri in year 2016.
 \item Ranked 2nd in College and 8th in overall \href {https://www.hackerearth.com/CodeStart2/leaderboard/}{leaderboard} at CodeStart 2.0 organised by IEEE-MSIT
 \item Ranked 94th in the \href {https://www.codechef.com/rankings/COOK76}{overall leaderboard} of the November Cook-Off on Codechef
 \item Second Runner­-up in Techvilla­ App Presentation held in Avensis - the Tech Fest of our college
 \item Active member of Codechef Discussion Forum. (Best Standing - 19)
 \item Active participation on online competitive coding websites - \href{http://www.codechef.com/users/nickzuck_007}{CodeChef}, \href{http://www.spoj.com/users/nickzuck\_007/}{Spoj}, \href{http://hackerrank.com/nickzuck\_007}{HackerRank}, \href {https://www.hackerearth.com/@nickzuck\_007}{HackerEarth}. Member Handle - nickzuck\_007
 \item Completed MOOC Courses Programming for Everybody(Python), CS101
 \item Moderator at Technocratz - the technical club of MSIT.
 \item Problem Curator  and Techincal Event Head in Code till You Die - a competitive programming event organized at techical fest of our college.
 \item Campus Ambassador at HackerEarth. 
\end{itemize}

%----------------------------------------------------------------------------------------
%	HOBBIES SECTION
%----------------------------------------------------------------------------------------

\section{Hobbies}
Competitive Coding, Watching TED talks, Chatting

\end{resume}
\end{document}